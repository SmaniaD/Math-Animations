% -*- TeX -*-
\documentclass[12pt]{amsart}
\usepackage[utf8]{inputenc}
\usepackage[T1]{fontenc}
\usepackage{geometry}
\geometry{margin=2.5cm}
\usepackage{hyperref}
\usepackage{xcolor}
\usepackage{listings}
\usepackage{enumitem}
\usepackage{fourier}
\hypersetup{
  colorlinks=true,
  linkcolor=blue,
  urlcolor=teal,
  pdfauthor={Daniel Smania},
  pdftitle={Wavelet Animation Script - User Manual}
}

% Code‑listing setup
\definecolor{bg}{RGB}{245,245,245}
\lstset{
  basicstyle=\ttfamily\small,
  backgroundcolor=\color{bg},
  frame=single,
  breaklines=true,
  columns=flexible,
  keywordstyle=\color{teal}\bfseries,
  commentstyle=\color{gray}\itshape,
  stringstyle=\color{purple},
  numbers=left,
  numberstyle=\tiny\color{gray},
  xleftmargin=1.5em,
  aboveskip=1em,
  belowskip=1em
}

\begin{document}

\begin{center}
{\LARGE\bfseries Associativity on $\pi_1$}\\ \ \\  { \LARGE\bfseries Daniel Smania} \\[0.5em]
Version: \texttt{1.0} \quad--\quad Last updated: \today \\

\end{center}
\vspace{1em}

This animation was created to explain why 
$$
\alpha \star (\beta \star \gamma) \simeq (\alpha \star \beta) \star \gamma
$$
in the fundamental group $\pi_1$, motivated by an answer we gave to a question on Mathematics Stack Exchange.

The path $\alpha \star (\beta \star \gamma)$ evolves as follows:
\begin{enumerate}[label=\textbf{\arabic*.}]
    \item During the first half of the interval $[0, 1/2]$, it travels along $\alpha$ at twice the original speed.

    \item During the second half $[1/2, 1]$, it moves along $\beta \star \gamma$. More precisely:
    \begin{enumerate}[label*=\arabic*.]
        \item On $[1/2, 3/4]$, it moves along $\beta$ at four times the original speed.
        \item On $[3/4, 1]$, it moves along $\gamma$ also at four times the original speed.
    \end{enumerate}
\end{enumerate}

On the other hand, the path $(\alpha \star \beta) \star \gamma$ evolves as follows:
\begin{enumerate}[label=\textbf{\arabic*.}]
    \item During the first half $[0, 1/2]$, it travels along $\alpha \star \beta$. In this interval:
    \begin{enumerate}[label*=\arabic*.]
        \item On $[0, 1/4]$, it moves along $\alpha$ at four times the original speed.
        \item On $[1/4, 1/2]$, it moves along $\beta$ also at four times the original speed.
    \end{enumerate}

    \item During the second half $[1/2, 1]$, it moves along $\gamma$ at twice the original speed.
\end{enumerate}
\vspace{1em}

Thus, in both cases the path moves through $\alpha$, $\beta$, and $\gamma$ in the same order, but with different speeds. The homotopy is constructed by continuously changing, with a parameter $s \in [0,1]$, the speeds at which we traverse each segment:

- At $s=0$, we move through $\alpha$, $\beta$, and $\gamma$ at speeds $2x$, $4x$, $4x$ respectively.

- At $s=1$, we move through them at speeds $4x$, $4x$, $2x$ respectively.

To be more precise:

- For fixed $s$, the map 
  $$
  \frac{4t}{2 - s}
  $$
  sends the interval $\left[0, \frac{2 - s}{4}\right]$ to $[0,1]$, with derivative $\frac{4}{2 - s}$. So at $s=0$ the derivative is $2$, and at $s=1$ it is $4$.

- For fixed $s$, the map 
  $$
  4t + s - 2
  $$
  sends $\left[\frac{2 - s}{4}, \frac{3 - s}{4}\right]$ to $[0,1]$, with constant derivative $4$.

- For fixed $s$, the map 
  $$
  \frac{4t + s - 3}{s + 1}
  $$
  sends $\left[\frac{3 - s}{4}, 1\right]$ to $[0,1]$, with derivative $\frac{4}{s + 1}$. At $s=0$ this is $4$, and at $s=1$ it is $2$.

\vspace{1em}

In other words, for every fixed $s$, the path $t \mapsto H(t,s)$ is just a reparametrization of the initial path $\omega = \alpha \star (\beta \star \gamma)$, moving at different "speeds". That is,
$$
H(t,s) = \omega(\theta_s(t)),
$$
where $\theta_s \colon [0,1] \to [0,1]$ is the homeomorphism defined by



$$
\theta_s(t) =
\begin{cases}
\displaystyle \frac{2t}{2 - s}, & t \in \left[0, \frac{2 - s}{4} \right], \\[1ex]
\displaystyle t + \frac{s}{4}, & t \in \left[ \frac{2 - s}{4}, \frac{3 - s}{4} \right], \\[1ex]
\displaystyle \frac{3}{4} + \frac{t + \frac{s}{4} - \frac{3}{4}}{s + 1}, & t \in \left[ \frac{3 - s}{4}, 1 \right].
\end{cases}
$$

The Python script generates an animation that shows the deformation given by $\theta_s$ as $s$ varies from $0$ to $1$.

\section*{License}
MIT License – Feel free to adapt and redistribute with attribution.

\end{document}

% -*- TeX -*-
\documentclass[12pt]{article}
\usepackage[utf8]{inputenc}
\usepackage[T1]{fontenc}
\usepackage{geometry}
\geometry{margin=2.5cm}
\usepackage{hyperref}
\usepackage{xcolor}
\usepackage{listings}
\usepackage{enumitem}
\usepackage{fourier}
\hypersetup{
  colorlinks=true,
  linkcolor=blue,
  urlcolor=teal,
  pdfauthor={Daniel Smania},
  pdftitle={Wavelet Animation Script - User Manual}
}

% Code‑listing setup
\definecolor{bg}{RGB}{245,245,245}
\lstset{
  basicstyle=\ttfamily\small,
  backgroundcolor=\color{bg},
  frame=single,
  breaklines=true,
  columns=flexible,
  keywordstyle=\color{teal}\bfseries,
  commentstyle=\color{gray}\itshape,
  stringstyle=\color{purple},
  numbers=left,
  numberstyle=\tiny\color{gray},
  xleftmargin=1.5em,
  aboveskip=1em,
  belowskip=1em
}

\begin{document}

\begin{center}
{\LARGE\bfseries Wavelet Animation Script User Manual}\\ \ \\  { \LARGE\bfseries Daniel Smania} \\[0.5em]
Version: \texttt{1.0} \quad--\quad Last updated: \today \\

\end{center}
\vspace{1em}

\tableofcontents
\newpage

\section{Introduction}
This script produces a frame-by-frame reconstruction of a one-dimensional function using discrete wavelet coefficients.\footnote{Source file: \texttt{wavelet.py}.} It is aimed at educators and researchers who wish to visualise how individual wavelets build up a signal.

Key features:
\begin{itemize}[nosep]
  \item Generates random functions of several types (smooth, discontinuous, etc.).
  \item Supports a wide range of discrete wavelets from PyWavelets.
  \item Saves high-quality GIF or MP4 animations (optional).
  \item Logs every execution for reproducibility (\texttt{wavelet\_history.log}).
\end{itemize}

\section{Prerequisites}
\begin{itemize}[nosep]
  \item \textbf{Python} $\ge$ 3.8
  \item Packages: \texttt{numpy}, \texttt{matplotlib}, \texttt{pywt} (PyWavelets), \texttt{argparse} (standard), \texttt{pillow}.
  \item To export MP4 you also need \href{https://ffmpeg.org/}{FFmpeg} in your \texttt{PATH}.
\end{itemize}

Install the scientific stack with:
\begin{lstlisting}[language=bash]
pip install numpy matplotlib pywavelets pillow

# Optional for MP4 export
sudo apt-get install ffmpeg   # Debian/Ubuntu
\end{lstlisting}

\section{Quick Start}
\begin{lstlisting}[language=bash]
python wavelet.py --wavelet_type db4 --function_type smooth_periodic --save gif
\end{lstlisting}
This command animates the first 64 \texttt{db4} wavelets reconstructing a periodic smooth signal and saves a GIF in the working directory.

\section{Command-Line Options}
\begin{description}[leftmargin=1.8cm, style=sameline]
  \item[\texttt{--frames\_per\_wavelet}] Frames allocated to each wavelet coefficient. Controls animation speed. Default: \texttt{12}.
  \item[\texttt{--wavelet\_type}] Wavelet family used by PyWavelets. Supported: \texttt{db4}, \texttt{db6}, \texttt{db8}, \texttt{haar}, \texttt{bior2.2}, \texttt{bior4.4}, \texttt{coif2}, \texttt{coif4}, \texttt{sym4}, \texttt{sym8}, \texttt{dmey}. Default: \texttt{haar}.
  \item[\texttt{--function\_type}] Type of synthetic function: \texttt{smooth}, \texttt{piecewise\_linear}, \texttt{discontinuous}, \texttt{smooth\_periodic}, \texttt{mix}. Default: \texttt{smooth\_periodic}.
  \item[\texttt{--function\_seed}] Integer seed for reproducible randomness. Default: \texttt{38324}.
  \item[\texttt{--number\_wavelets}] Maximum number of coefficients to animate. Default: \texttt{64}.
  \item[\texttt{--save}] Format for saving animation: \texttt{gif}, \texttt{mp4}, or \texttt{none}. Default: \texttt{none}.
\end{description}

\section{Output Files}
\subsection*{Animation}
When \texttt{--save} is \texttt{gif} or \texttt{mp4}, the script writes a file named:
\begin{center}
\texttt{wavelet-<YYYY-MM-DD-HH-MM-SS>-<wavelet>\_<func>\_<N>w\_<seed>s\_<F>f.(gif|mp4)}
\end{center}

\subsection*{Run History}
Each command is logged to \texttt{wavelet\_history.log}. Only the 100 most recent entries are kept.

\section{Examples}
\subsection*{1. Fast preview without saving}
\begin{lstlisting}[language=bash]
python wavelet.py --wavelet_type sym4 --frames_per_wavelet 5 --number_wavelets 32
\end{lstlisting}

\subsection*{2. High-quality MP4}
\begin{lstlisting}[language=bash]
python wavelet.py \
  --wavelet_type bior4.4 \
  --function_type discontinuous \
  --frames_per_wavelet 15 \
  --number_wavelets 120 \
  --save mp4
\end{lstlisting}

\section{Workflow Internals}
\begin{enumerate}[label=\arabic*.]
  \item Generate input signal $f$ using \texttt{random\_function()}.
  \item Compute wavelet decomposition via \texttt{pywt.wavedec}.
  \item For each animation frame:
    \begin{enumerate}[nosep]
      \item Build partial coefficient list with added wavelets.
      \item Reconstruct $f_{\\text{partial}}$.
      \item Overlay orange fill (wavelet being added) and blue fill (cumulative).
    \end{enumerate}
  \item Optionally export via Pillow (GIF) or FFmpeg (MP4).
\end{enumerate}

\section{Troubleshooting}
\begin{description}[leftmargin=1.8cm, style=sameline]
  \item[Missing FFmpeg] Ensure \texttt{ffmpeg} is installed and in your \texttt{PATH}.
  \item[Large GIF files] Increase \texttt{frames\_per\_wavelet} or reduce \texttt{number\_wavelets}.
  \item[Memory errors] Reduce \texttt{dpi} or figure size in the script.
\end{description}

\section{Extending the Script}
\begin{itemize}[nosep]
  \item Add new function types by editing \texttt{random\_function()}.
  \item Try other PyWavelets wavelet families.
  \item Replace generated signal with real data arrays.
\end{itemize}

\section*{License}
MIT License – Feel free to adapt and redistribute with attribution.

\end{document}
